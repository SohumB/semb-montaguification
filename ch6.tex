\def\swap{\leftrightarrow}
\def\peq{\boxplus}
\def\teq{\otimes}
\def\xor{\oplus}
\def\nimplies{\triangleright}
\def\from{\gets}
\def\real{\mathbb{R}}
\def\rtnl{\mathbb{Q}}
\def\cplx{\mathbb{C}}
\def\setz{\mathbb{Z}}
\def\exp{\mathbb{E}}
\def\f{\mathbb{F}}
\def\N{\mathcal{N}}
\def\E{\mathbb{E}}
\def\zplus{\setz^{+}}
\def\nat{\mathbb{N}}
\def\transpose{^\top}
\def\kleene{^{*}}
\def\const{\mathtt{C}}
\def\p{\partial}
\def\gl{{GL}}
\def\sll{{SL}}
\def\Ub{\mathbf{U}\ }
\def\Sb{\mathbf{S}\ }
\def\vt{\vartheta}
\documentclass[a4paper]{article}
\renewcommand{\thesection}{\arabic{section}}
\renewcommand{\thesubsection}{\alph{subsection}.}
\renewcommand{\labelenumi}{(\alph{enumi})}
\renewcommand{\labelenumii}{(\alph{enumii})}
\renewcommand{\thefootnote}{\fnsymbol{footnote}}
\renewcommand{\thempfootnote}{\fnsymbol{mpfootnote}}
\usepackage{amsmath,amssymb,amsthm}
\usepackage{euler,beton}
\usepackage{algorithmic}
\usepackage{fontspec}
\usepackage[agsm]{harvard}
\usepackage{hyperref}
\usepackage{xunicode,xltxtra}
\usepackage{graphicx}
\usepackage{sectsty}
\usepackage{multicol}
\usepackage{semantic}
%\usepackage{listings}
%\usepackage{verbatim}
\usepackage{color}
%\usepackage{pgf,tikz}
\usepackage{booktabs}
\usepackage{fullpage}%\usepackage[left=3cm,right=3cm,top=2.5cm,bottom=2.5cm]{geometry}
\usepackage{fancyhdr}
%\usetikzlibrary{trees,arrows,automata,shapes,positioning,calc}
\pagestyle{fancy}
\fancyhf{}
\renewcommand{\headrulewidth}{0pt}
\lfoot{}
\cfoot{\thepage}
\rfoot{Semantics B \\ Chapter 6}
\setmainfont[Ligatures={Common},ItalicFont = Linux Libertine O Italic]{Linux Libertine O}
\providecommand{\abs}[1]{\left\lvert#1\right\rvert}
\providecommand{\norm}[1]{\left\lVert#1\right\rVert}
\providecommand{\tuple}[1]{\left\langle#1\right\rangle}
\providecommand{\max}[1]{\max\left(#1\right)}
\providecommand{\min}[1]{\min\left(#1\right)}
\providecommand{\gen}[1]{\left\langle#1\right\rangle}
\providecommand{\coltwo}[2]{\begin{bmatrix}#1 \\ #2\end{bmatrix}}
\providecommand{\msfour}[4]{\begin{bmatrix}#1 & #2 \\ #3 & #4\end{bmatrix}}
\providecommand{\msnine}[9]{\begin{bmatrix}#1 & #2 & #3 \\ #4 & #5 & #6 \\ #7 & #8 & #9\end{bmatrix}}
\providecommand{\deduces}{\mathbin{\vdash}}
\providecommand{\ddeduces}{\mathbin{\vdash\!\!\!\dashv}}
\providecommand{\sch}[1]{\mathcal{#1}}
\providecommand{\dimplies}{\Leftrightarrow}
\providecommand{\ceil}[1]{\left\lceil#1\right\rceil}
\providecommand{\floor}[1]{\left\lfloor#1\right\rfloor}
\renewcommand{\and}{\mathbin{\&}}
\renewcommand{\implies}{\Rightarrow}
\renewcommand{\impliedby}{\Leftarrow}
\DeclareMathOperator{\sech}{sech}
\DeclareMathOperator{\adj}{adj}
\DeclareMathOperator{\infi}{inf}
\DeclareMathOperator{\supre}{sup}
\DeclareMathOperator{\round}{round}
\DeclareMathOperator{\pascal}{pascal}
\DeclareMathOperator{\dom}{domain}
\DeclareMathOperator{\st}{s.t }
\DeclareMathOperator{\argmax}{arg\ max}
\providecommand{\Resc}{\Res\kleene}
\newtheorem{lemma}{Lemma}
\newtheorem{sublemma}{Sublemma}
\sectionfont{\pagebreak[3]}
\subsectionfont{\pagebreak[2]}
\reservestyle{\csyms}{\mathcal}
\csyms{A,B,C,D}
\mathlig{|}{\mid}
\mathlig{==}{\equiv}
\mathlig{~>}{\leadsto}
\mathlig{~=}{\approx}
\mathlig{~}{\lnot}
\mathlig{|->}{\mapsto}
\mathlig{||-}{\Vdash}
\mathlig{<F>}{\langle{}F \rangle}
\mathlig{|/=}{\nvDash}

\title{H\&K Chapter 6}
\author{Sohum Banerjea, Andrew Hedding, Lisa Hofmann}
\date{\today}

\usepackage{qtree}

\begin{document}

\maketitle

\section{Quantifiers vs. Proper Names}

\begin{center}
  \begin{tabular}{c c}
  \Tree [.t [.{(et)t} \emph{nothing} ] [.et \emph{vanished} ] ]
   &
  \Tree [.t [.e \emph{Mary} ] [.et \emph{vanished} ] ]
   \\
  \end{tabular}
\end{center}

\begin{itemize}
  \item \emph{nothing}  $\rightsquigarrow \lambda P_{et}. \neg\exists x_e.P(x)$
  \item \emph{Ann}    $\rightsquigarrow \textsc{Ann}_e$ (/$\textsc{a}_e$)
  \item \emph{vanish}   $\rightsquigarrow \textsc{vanish}_{et}$
\end{itemize}

\subsection{Quantifiers aren't of type $e$}

\begin{itemize}
  \item Not all quantifiers are upward monotonic ($P \land Q \rightarrow P$)
  \begin{itemize}
    \item John came yesterday morning. $\Rightarrow$ John came yesterday.
    \item $P(x_e) \land Q_(x_e) \rightarrow P(x_e)$
    \item No letter came yesterday morning. $\not\Rightarrow$ No letter came yesterday.
    \item $\neg\exists x_e.P(x) \land Q(x) \not\rightarrow \neg\exists x_e.P(x)$
    \item Entailment from a more specific predication (subset) to a more general predication (superset) is not necessarily given under quantification.
    \item Quantifiers like \lq at most one\rq\ and \lq no\rq\ are \emph{downward entailing}.
  \end{itemize}

  \item Not all quantifiers obey the law of contradiction ($\neg P \land \neg P$)
  \begin{itemize}
    \item Mt. Rainier is on this side of the border and Mt. Rainier is on the other side of the border. $\Leftrightarrow \bot$
    \item $P(x_e) \land Q(x_e) \leftrightarrow \bot,$ where $P^\rightsquigarrow \cap Q^\rightsquigarrow = \varnothing$
    \item Some mountains are on this side of the border and some mountains are on the other side of the border.
    \item $\exists x_e.P(x) \land Q(x) \not \leftrightarrow \bot,$ even if $P^\rightsquigarrow \cap Q^\rightsquigarrow = \varnothing$
  \end{itemize}

  \item Not all quantifiers obey the law of the excluded middle ($P \lor \neg P$)

  \begin{itemize}
    \item
  \end{itemize}

  \item Scope ambiguities
\end{itemize}

\subsection{Quantifiers aren't of type $et$}

\begin{itemize}
  \item Should also be upward entailing
  \item Contradiction, Excluded middle + superset entailment should still hold
\end{itemize}


\section{Semantics of quantifiers}

\subsection{Compositional semantics}
\begin{itemize}
\item Consider an expanded, compositional version of the tree from before.
  \begin{center}
  \begin{tabular}{c}
    \Tree [.t [.{(et)t} [.{???} \emph{no} ] [.et \emph{thing} ]] [.et \emph{vanished} ] ]
  \end{tabular}
\end{center}

\item \emph{no}, \emph{every}, and \emph{some} need to have type  $(et)(et)t$.
\item \emph{every} $~> \lambda P_{et}\ \lambda Q_{et}\ \forall x_e\ P(x) -> Q(x)$
\item \emph{some} $~> \lambda P_{et}\ \lambda Q_{et}\ \exists x_e\ P(x) \land Q(x)$
\item \emph{no} $~> \lambda P_{et}\ \lambda Q_{et}\ \lnot\exists x_e\ P(x) \land Q(x)$
\end{itemize}

\subsection{Relations between sets}

\begin{itemize}
\item Each of these are translations into a characteristic function notation of set relations
\item If $R$ is a relation between sets, we can produce the function version as follows
  $$f_R = \lambda P_{et}\ \lambda Q_{et}\ \{\{x\mid P(x) = \top\},\{x\mid Q(x) = \top\}\}\in R$$
\item
\end{itemize}


\section{Presuppositional behaviour of quantifiers}



\end{document}

%%% Local Variables:
%%% mode: xetex
%%% TeX-master: t
%%% TeX-engine: xetex
%%% End:
