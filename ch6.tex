\def\swap{\leftrightarrow}
\def\peq{\boxplus}
\def\teq{\otimes}
\def\xor{\oplus}
\def\nimplies{\triangleright}
\def\from{\gets}
\def\real{\mathbb{R}}
\def\rtnl{\mathbb{Q}}
\def\cplx{\mathbb{C}}
\def\setz{\mathbb{Z}}
\def\exp{\mathbb{E}}
\def\f{\mathbb{F}}
\def\N{\mathcal{N}}
\def\E{\mathbb{E}}
\def\zplus{\setz^{+}}
\def\nat{\mathbb{N}}
\def\transpose{^\top}
\def\kleene{^{*}}
\def\const{\mathtt{C}}
\def\p{\partial}
\def\gl{{GL}}
\def\sll{{SL}}
\def\Ub{\mathbf{U}\ }
\def\Sb{\mathbf{S}\ }
\def\vt{\vartheta}
\documentclass[a4paper]{article}
\renewcommand{\thesection}{\arabic{section}}
\renewcommand{\thesubsection}{\alph{subsection}.}
\renewcommand{\labelenumi}{(\alph{enumi})}
\renewcommand{\labelenumii}{(\alph{enumii})}
\renewcommand{\thefootnote}{\fnsymbol{footnote}}
\renewcommand{\thempfootnote}{\fnsymbol{mpfootnote}}
\usepackage{amsmath,amssymb,amsthm}
\usepackage{euler,beton}
\usepackage{algorithmic}
\usepackage{fontspec}
\usepackage[agsm]{harvard}
\usepackage{hyperref}
\usepackage{xunicode,xltxtra}
\usepackage{graphicx}
\usepackage{sectsty}
\usepackage{multicol}
\usepackage{semantic}
%\usepackage{listings}
%\usepackage{verbatim}
\usepackage{color}
%\usepackage{pgf,tikz}
\usepackage{booktabs}
\usepackage{fullpage}%\usepackage[left=3cm,right=3cm,top=2.5cm,bottom=2.5cm]{geometry}
\usepackage{fancyhdr}
%\usetikzlibrary{trees,arrows,automata,shapes,positioning,calc}
\pagestyle{fancy}
\fancyhf{}
\renewcommand{\headrulewidth}{0pt}
\lfoot{}
\cfoot{\thepage}
\rfoot{Semantics B \\ Chapter 6}
\setmainfont[Ligatures={Common},ItalicFont = Linux Libertine O Italic]{Linux Libertine O}
\providecommand{\abs}[1]{\left\lvert#1\right\rvert}
\providecommand{\norm}[1]{\left\lVert#1\right\rVert}
\providecommand{\tuple}[1]{\left\langle#1\right\rangle}
\providecommand{\max}[1]{\max\left(#1\right)}
\providecommand{\min}[1]{\min\left(#1\right)}
\providecommand{\gen}[1]{\left\langle#1\right\rangle}
\providecommand{\coltwo}[2]{\begin{bmatrix}#1 \\ #2\end{bmatrix}}
\providecommand{\msfour}[4]{\begin{bmatrix}#1 & #2 \\ #3 & #4\end{bmatrix}}
\providecommand{\msnine}[9]{\begin{bmatrix}#1 & #2 & #3 \\ #4 & #5 & #6 \\ #7 & #8 & #9\end{bmatrix}}
\providecommand{\deduces}{\mathbin{\vdash}}
\providecommand{\ddeduces}{\mathbin{\vdash\!\!\!\dashv}}
\providecommand{\sch}[1]{\mathcal{#1}}
\providecommand{\dimplies}{\Leftrightarrow}
\providecommand{\ceil}[1]{\left\lceil#1\right\rceil}
\providecommand{\floor}[1]{\left\lfloor#1\right\rfloor}
\renewcommand{\and}{\mathbin{\&}}
\renewcommand{\implies}{\Rightarrow}
\renewcommand{\impliedby}{\Leftarrow}
\DeclareMathOperator{\sech}{sech}
\DeclareMathOperator{\adj}{adj}
\DeclareMathOperator{\infi}{inf}
\DeclareMathOperator{\supre}{sup}
\DeclareMathOperator{\round}{round}
\DeclareMathOperator{\pascal}{pascal}
\DeclareMathOperator{\dom}{domain}
\DeclareMathOperator{\st}{s.t }
\DeclareMathOperator{\argmax}{arg\ max}
\providecommand{\Resc}{\Res\kleene}
\newtheorem{lemma}{Lemma}
\newtheorem{sublemma}{Sublemma}
\sectionfont{\pagebreak[3]}
\subsectionfont{\pagebreak[2]}
\reservestyle{\csyms}{\mathcal}
\csyms{A,B,C,D}
\mathlig{|}{\mid}
\mathlig{==}{\equiv}
\mathlig{~>}{\rightsquigarrow}
\mathlig{~=}{\approx}
\mathlig{~}{\lnot}
\mathlig{|->}{\mapsto}
\mathlig{||-}{\Vdash}
\mathlig{<F>}{\langle{}F \rangle}
\mathlig{|/=}{\nvDash}

\title{H\&K Chapter 6}
\author{Sohum Banerjea, Andrew Hedding, Lisa Hofmann}
\date{\today}

\usepackage{qtree}

\begin{document}

\maketitle

\section{Quantifiers vs. Proper Names}

\begin{center}
  \begin{tabular}{c c}
  \Tree [.t [.{(et)t} \emph{nothing} ] [.et \emph{vanished} ] ]
   &
  \Tree [.t [.e \emph{Ann} ] [.et \emph{vanished} ] ]
   \\
  \end{tabular}
\end{center}

\begin{itemize}
  \item \emph{nothing}  $\rightsquigarrow \lambda P_{et}. \neg\exists x_e.P(x)$
  \item \emph{Ann}    $\rightsquigarrow \textsc{Ann}_e$ (/$\textsc{a}_e$)
  \item \emph{vanish}   $\rightsquigarrow \textsc{vanish}_{et}$
\end{itemize}

\subsection{Quantifiers aren't of type $e$}

\begin{itemize}

  \item Not all quantifiers are upward monotonic ($P \land Q \rightarrow P$)
  \begin{itemize}
    \item John came yesterday morning. $\Rightarrow$ John came yesterday.
    \item $P(x_e) \land Q_(x_e) \rightarrow P(x_e)$
    \item No letter came yesterday morning. $\not\Rightarrow$ No letter came yesterday.
    \item $\neg\exists x_e.P(x) \land Q(x) \not\rightarrow \neg\exists x_e.P(x)$
    \item Entailment from a more specific predication (subset) to a more general predication (superset) is not necessarily given under quantification.
    \item Quantifiers like \lq at most one\rq\ and \lq no\rq\ are \emph{downward entailing}.
  \end{itemize}

  \item Not all quantifiers obey the law of contradiction ($\neg (P \land \neg P)$)
  \begin{itemize}
    \item Mt. Rainier is on this side of the border and Mt. Rainier is on the other side of the border. $\Leftrightarrow \bot$
    \item $P(x_e) \land Q(x_e) \leftrightarrow \bot,$ where $P^\rightsquigarrow \cap Q^\rightsquigarrow = \varnothing$
    \item Some mountains are on this side of the border and some mountains are on the other side of the border.
    \item $\exists x_e.P(x) \land Q(x) \not \leftrightarrow \bot,$ even if $P^\rightsquigarrow \cap Q^\rightsquigarrow = \varnothing$
  \end{itemize}

  \item Not all quantifiers obey the law of the excluded middle ($P \lor \neg P$)
  \begin{itemize}
  	\item I am over 30 years old or I am under 30 years old. $\Leftrightarrow \top$
    \item $P(x_e) \lor \neg P(x_e) \leftrightarrow \top$
    \item Every woman in this room is over 30 years old or every woman in this room is under 30 years old. $\not \Leftrightarrow \top$
    \item $\forall x_e.(P(x)) \land \forall x_e.(Q(x)) \not\leftrightarrow \top$
  \end{itemize}

  \item Scope ambiguities
  \begin{itemize}
  	\item Ann saw a fly.
  	\item Every frog saw a fly. ($\forall > \exists$) / ($\exists > \forall$)
  	\item It didn't snow on christmas day.
  	\item It didn't snow on more than two of these days. ($\exists > \neg$) / ($\neg > \exists$)
  \end{itemize}

\end{itemize}

\subsection{Quantifiers aren't of type $et$}

\begin{itemize}
  \item Suggestion: lift the type only a little bit and have quantifiers denote sets of individuals
  \item have DPs of type $et$ and verbs of type $(et)t$
  \item \emph{Ann} $~> \lambda x_e.x = \textsc{a}_e$\\
        \emph{everything} $~> \lambda x_e.1$\\
        \emph{nothing} $~> \lambda x_e.0$\\
        % \emph{something} $~> \lambda x_e.\exists P_{et}.$
        \emph{vanished} $~> \lambda P_{et}.\forall x_e.P(x) \rightarrow \textsc{vanish}_{et}(x)$
  \item Problems with \emph{nothing}:
  \begin{itemize}
    \item It works for upward entailing quantifiers like \emph{everything}:\\
    \Tree [.{$[\lambda P_{et}.\forall x_e.P(x) \rightarrow \textsc{vanish}_{et}(x)](\lambda x_e.1) \Leftrightarrow \forall x_e.1 \rightarrow \textsc{vanish}_{et}(x) \Leftrightarrow$ \textcolor{green}{$\forall x_e. \textsc{vanish}_{et}(x)$}}
      [.{$\lambda x_e.1$} \emph{everything} ]
      [.{$\lambda P_{et}.\forall x_e.P(x) \rightarrow \textsc{vanish}_{et}(x)$} \emph{vanished} ]
      ]
    \item It doesn't work for downward entailing quantifiers like \emph{nothing:}\\
    \Tree [.{$[\lambda P_{et}.\forall x_e.P(x) \rightarrow \textsc{vanish}_{et}(x)](\lambda x_e.0)
    \Leftrightarrow \forall x_e.0 \rightarrow \textsc{vanish}_{et}(x)
    \Leftrightarrow$ \textcolor{red}{$\forall x_e.1$}}
      [.{$\lambda x_e.0$} \emph{nothing} ]
      [.{$\lambda P_{et}.\forall x_e.P(x) \rightarrow \textsc{vanish}_{et}(x)$} \emph{vanished} ]
    ]
    \item Because the empty set is a subset of every set, a sentence of the form \emph{\lq Nothing} V\emph{ed\rq} (for any verb V) would be trivially true.
  \end{itemize}
\end{itemize}


\section{Semantics of quantifiers}

\subsection{Compositional semantics}
\begin{itemize}
\item Consider an expanded, compositional version of the tree from before.
  \begin{center}
  \begin{tabular}{c}
    \Tree [.t [.{(et)t} [.{???} \emph{no} ] [.et \emph{thing} ]] [.et \emph{vanished} ] ]
  \end{tabular}
\end{center}

\item \emph{no}, \emph{every}, and \emph{some} need to have type  $(et)(et)t$.
\item \emph{every} $~> \lambda P_{et}.\ \lambda Q_{et}.\ \forall x_e.\ P(x) -> Q(x)$
\item \emph{some} $~> \lambda P_{et}.\ \lambda Q_{et}.\ \exists x_e.\ P(x) \land Q(x)$
\item \emph{no} $~> \lambda P_{et}.\ \lambda Q_{et}.\ \lnot\exists x_e.\ P(x) \land Q(x)$
\end{itemize}

\subsection{Relations between sets}

\begin{itemize}
\item Each of these are translations into a characteristic function notation of set relations
\item If $R$ is a relation between sets, we can produce the function version as follows
  $$f_R = \lambda P_{et}.\ \lambda Q_{et}.\ \{\{x\mid P(x) = \top\},\{x\mid Q(x) = \top\}\}\in R$$
\item Set relations have certain formal properties, which we can translate into the function notation
  \begin{itemize}
  \item $f$ is reflexive iff $\forall a_{et}.\ f(a)(a)$
  \item $f$ is irreflexive iff $\forall a_{et}.\ \lnot f(a)(a)$
  \item $f$ is symmetric iff $\forall a_{et}, b_{et}.\ f(a)(b) \iff f(b)(a)$
  \item $f$ is antisymmetric iff $\forall a_{et}, b_{et}.\ f(a)(b) \land f(b)(a) -> a = b$
  \item $f$ is transitive iff $\forall a_{et}, b_{et}, c_{et}.\ f(b)(a) \land f(c)(b) -> f(c)(a)$
  \item $f$ is conservative iff $\forall a_{et}, b_{et}.\ f(b)(a) \iff f(\lambda x.\ a(x) \land b(x))(a)$
  \item $f$ is intersective iff $\exists g.\ f(a_{et})(b_{et}) = g(size(\lambda x_e.\ a(x) \land b(x)))$
  \item $f$ is left upward monotone iff $\forall a_{et}, b_{et}, c_{et}.\ \Big(f(c)(a) \land \forall x_e.\ a(x)
    -> b(x)\Big) -> f(c)(b)$
  \item $f$ is left downward monotone iff $\forall a_{et}, b_{et}, c_{et}.\ \Big(f(c)(b) \land \forall x_e.\ a(x)
    -> b(x)\Big) -> f(c)(a)$
  \item $f$ is right upward monotone iff $\forall a_{et}, b_{et}, c_{et}.\ \Big(f(a)(c) \land \forall x_e.\ a(x)
    -> b(x)\Big) -> f(b)(c)$
  \item $f$ is right downward monotone iff $\forall a_{et}, b_{et}, c_{et}.\ \Big(f(b)(c) \land \forall x_e.\ a(x)
    -> b(x)\Big) -> f(a)(c)$
  \end{itemize}
\item There-insertion correlates with intersectivity:
  \begin{itemize}
  \item There are few apples in my pocket: $g$ compares to a small threshold
  \item There are some apples in my pocket: $g$ compares to a larger threshold
  \item There are many apples in my pocket: $g$ compares to an even larger threshold
  \item There are no apples in my pocket: $g = \lambda x_\N. x == 0$
  \item There are exactly two apples in my pocket: $g = \lambda x_\N. x == 2$
  \item * There are every apple in my pocket: $g$ has no access to the size of the first set
  \item * There are almost every apple in my pocket: ditto
  \item * There are most apples in my pocket: ditto
  \item * There are not every apple in my pocket: you need the size of $a(x) \land \lnot b(x)$
  \end{itemize}
\item Negative polarity cases \emph{mostly} seems to correlate with downward monotonicity in the target,
  though a lot of these judgements are unstable. There are also \textcolor{red}{counterexamples}.
  \begin{itemize}
  \item (Assume for all these cases that all apples are fruits.)
  \item No person who has ever lived [\ldots]: if no fruits are red, then no apples are red
  \item No person ever has lived: if no red things are fruits, then no red things are apples
  \item Few people who have ever lived [\ldots]: if few fruits are red, then few apples are red
  \item Few people have ever lived: if few red things are fruits, then few red things are apples
  \item Every person who has ever lived [\ldots]: if every fruit is red, then every apple is red
  \item * Every person has ever lived: * if every red thing is a fruit, then every red thing is an apple
  \item * Some people who have ever lived [\ldots]: * if some fruits are red, then some apples are red
  \item * Some people have ever lived: * if some red things are fruits, then some red things are
    apples
  \item * Many people who have ever lived [\ldots]: * if many fruits are red, then many apples are red
  \item * Many people have ever lived: * if many red things are fruits, then many red things are
    apples
  \item \textcolor{red}{Almost every person who has ever lived [\ldots]: * if almost every fruit is red, then
      almost every apple is red}
  \item * Almost every person has ever lived: * if almost every red thing is a fruit, then almost
    every red thing is an apple
  \item \textcolor{red}{Most people who have ever lived [\ldots]: * if most fruits are red, then most apples are red}
  \item * Most people have ever lived: * if most red things are fruits, then most red things are
    apples
  \item * Not every person who has ever lived [\ldots]: * if not every fruit is red, then not every apple is red
  \item \textcolor{red}{* Not every person has ever lived: if not every red thing is a fruit, then not every
      red thing is an apple}
  \item * Exactly two people who have ever lived [\ldots]: * if exactly two fruits are red, then exactly two
    apples are red
  \item \textcolor{red}{Exactly two people have ever lived: * if exactly two red things are fruits, then exactly two red
    things are apples}
  \end{itemize}
\end{itemize}


\section{Presuppositional behavior of quantifiers}
\begin{itemize}
\item Heim and Kratzer have assumed thus far that all quantifiers denote \emph{total functions}. That is, any quantifier $Q_{(et)(et)t}$ applied to arguments $\alpha_{et}$ and $\beta_{et}$, will always return $\lbrace1, 0\rbrace$. For example:
\begin{itemize}
\item $\forall x_e.\ (\textsc{cat}_{et}(x) -> \textsc{purr}_{et}(x))$ \emph{Every cat purred} will be true in the case that all cats in the domain purred, and false otherwise. In the case that there are no cats, the conditional is vacuously satisfied.
\end{itemize}
\item As Heim and Kratzer note, however, the claim that all quantifiers are total functions predicts that they will not give rise to presupposition failures.
\item In fact, there is evidence that many speakers interpret quantifiers as presuppositional triggers.
\end{itemize}
\subsection{Some quantifiers cause presuppositions}
\begin{itemize}
\item There seem to be clear cases where the `standard' analysis (i.e.\ total functions) of quantifiers fails to correctly correspond to speaker intuitions.
\begin{itemize}
\item \emph{Every American king lives in New York City}. The total function analysis of quantifiers predicts this sentence to be true due to the fact that the restrictor of the quantifier is the empty set.
\item \emph{I didn't see both cats}. This sentence is predicted to be true in a situation where there is only one cat and I saw it.
\end{itemize}
\item In both cases, most speakers will interpret these sentences as presupposition failures, rather than true.
\item \textbf{Presuppositionality Hypothesis}: In natural languages, all lexical items with denotations of type $(et)(et)t$ are presuppositional.
\item Formally speaking, this requires that the restrictor of the quantifier be non-empty:
  $$\lambda P_{et}.\ \lambda P^\prime_{et}.\ \Big(\exists x_e.\ P(x)\Big) \land \Big(\forall x_e.\ P(x) -> P^\prime(x)\Big)$$
\item Additionally, the presuppositional analysis avoids the \emph{Samaritan Paradox}:
  \begin{enumerate}
  \item \emph{The town regulations require that there be no trespassing.}
  \item \emph{The town regulations require that all trespassers be fined.}
  \item \emph{The town regulations require that no trespassers be fined.}
  \end{enumerate}
\item Intuitively, (a) and (b) could simultaneously be true when (c) is false. However, the `standard' analysis predicts that (b) and (c) will both be true in those worlds that are compatible with (a). Because there are no actual trespassers in the worlds compatible with the (a), quantifying over the trespassers in (b) and (c) will make the sentences vacuously true.
\end{itemize}
\subsection{Cases where quantifiers don't clearly trigger presuppositions}
\begin{itemize}
\item There are cases, however, where speaker judgements vary regarding the presuppositions of quantifiers:
\begin{itemize}
\item \emph{No American king lived in New York City}
\item \emph{Two American kings lived in New York City}
\end{itemize}
\item About half of speakers judge both sentences to be presuppositional failures and half judge the former true and the latter false.
\item Similarly, many speakers judge the following sentences to be true, even though the presuppositional analysis predicts that they will be presuppositional failures:
\begin{itemize}
\item \emph{Every unicorn has one horn.}
\item \emph{Every unicorn is a unicorn.}
\end{itemize}
\item \textbf{Potential Solution}: The quantifier only presupposes that that a unicorn is `mythologically possible,' not that it exists in the real world.
\item \emph{Two UFOs landed in my backyard yesterday}. Assuming there are no UFOs, the Presuppositionality Hypothesis predicts that this sentence will give rise to a presuppositional failure. Although it can only be uttered sincerely by someone who believes that there are UFOs, it seems intuitively that it will be understood as false, rather than a presupposition failure.
\end{itemize}
\subsection{Strong and Weak Determiners}
\begin{itemize}
\item Heim and Kratzer note that quantifiers can be split into two categories which roughly correspond to their presuppositional characteristics.
\begin{itemize}
\item \textbf{Strong Determiners:} \emph{every, almost every, not every, most, both and neither}
\item \textbf{Weak Determiners:} \emph{no, a, few, many and numerals}
\end{itemize}
\item Strong determiners tend to give rise to presupposition failures and weak determiners to mixed judgements.
\begin{itemize}
	\item \emph{If you find \underline{every} mistake I'll give you a fine reward.}
\begin{itemize}
\item \textsc{presupposition:} I believe there are mistakes.
\end{itemize}
	\item \emph{If you find \underline{a} mistake I'll give you a fine reward.}
\end{itemize}
\item Strong/weak distinction also determines whether the quantifier can be used in \emph{there be} constructions:
\begin{itemize}
\item \emph{There are no cats that I don't love.}
\item \emph{There are many cats that I don't love.}
\item * \emph{There is every cat that I don't love.}
\item * \emph{There are most cats that I don't love.}
\end{itemize}
\item While the strong/weak distinction sheds light on the presuppositional nature of quantifiers, pragmatic and grammatical factor also seems to play a role. For example:
  \begin{enumerate}
  \item \emph{No American king lived in New York City.}
    \begin{itemize}
    \item \textsc{presupposition:} There have been American kings.
    \end{itemize}
  \item \emph{There were no American kings that lived in New York City.}
  \end{enumerate}
\item Most speakers that judge (a) as a presupposition failure will judge (b) as true.
\item It is an open question as to how the presuppositional tendencies of quantifiers should be modeled and formalized.
\end{itemize}


\end{document}

%%% Local Variables:
%%% mode: xetex
%%% TeX-master: t
%%% TeX-engine: xetex
%%% End:
