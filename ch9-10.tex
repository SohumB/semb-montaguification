\def\swap{\leftrightarrow}
\def\peq{\boxplus}
\def\teq{\otimes}
\def\xor{\oplus}
\def\nimplies{\triangleright}
\def\from{\gets}
\def\real{\mathbb{R}}
\def\rtnl{\mathbb{Q}}
\def\cplx{\mathbb{C}}
\def\setz{\mathbb{Z}}
\def\exp{\mathbb{E}}
\def\f{\mathbb{F}}
\def\N{\mathcal{N}}
\def\E{\mathbb{E}}
\def\zplus{\setz^{+}}
\def\nat{\mathbb{N}}
\def\transpose{^\top}
\def\kleene{^{*}}
\def\const{\mathtt{C}}
\def\p{\partial}
\def\gl{{GL}}
\def\sll{{SL}}
\def\Ub{\mathbf{U}\ }
\def\Sb{\mathbf{S}\ }
\def\vt{\vartheta}
\documentclass[a4paper]{article}
\renewcommand{\thesection}{\arabic{section}}
\renewcommand{\thesubsection}{\alph{subsection}.}
\renewcommand{\labelenumi}{(\alph{enumi})}
\renewcommand{\labelenumii}{(\alph{enumii})}
\renewcommand{\thefootnote}{\fnsymbol{footnote}}
\renewcommand{\thempfootnote}{\fnsymbol{mpfootnote}}
\usepackage{amsmath,amssymb,amsthm}
\usepackage{euler,beton}
\usepackage{algorithmic}
\usepackage{fontspec}
\usepackage[agsm]{harvard}
\usepackage{hyperref}
\usepackage{xunicode,xltxtra}
\usepackage{graphicx}
\usepackage{sectsty}
\usepackage{multicol}
\usepackage{stmaryrd}
\usepackage{semantic}
\usepackage{qtree}
\usepackage{tikz-qtree-compat}
%\usepackage{listings}
%\usepackage{verbatim}
\usepackage{color}
%\usepackage{pgf,tikz}
\usepackage{booktabs}
\usepackage{fullpage}%\usepackage[left=3cm,right=3cm,top=2.5cm,bottom=2.5cm]{geometry}
\usepackage{fancyhdr}
\usepackage{qtree}
%\usetikzlibrary{trees,arrows,automata,shapes,positioning,calc}
\pagestyle{fancy}
\fancyhf{}
\renewcommand{\headrulewidth}{0pt}
\lfoot{}
\cfoot{\thepage}
\rfoot{Semantics B \\ Chapter 6}
\setmainfont[Ligatures={Common},ItalicFont = Linux Libertine O Italic]{Linux Libertine O}
\providecommand{\abs}[1]{\left\lvert#1\right\rvert}
\providecommand{\norm}[1]{\left\lVert#1\right\rVert}
\providecommand{\tuple}[1]{\left\langle#1\right\rangle}
\providecommand{\max}[1]{\max\left(#1\right)}
\providecommand{\min}[1]{\min\left(#1\right)}
\providecommand{\gen}[1]{\left\langle#1\right\rangle}
\providecommand{\coltwo}[2]{\begin{bmatrix}#1 \\ #2\end{bmatrix}}
\providecommand{\msfour}[4]{\begin{bmatrix}#1 & #2 \\ #3 & #4\end{bmatrix}}
\providecommand{\msnine}[9]{\begin{bmatrix}#1 & #2 & #3 \\ #4 & #5 & #6 \\ #7 & #8 & #9\end{bmatrix}}
\providecommand{\deduces}{\mathbin{\vdash}}
\providecommand{\ddeduces}{\mathbin{\vdash\!\!\!\dashv}}
\providecommand{\sch}[1]{\mathcal{#1}}
\providecommand{\dimplies}{\Leftrightarrow}
\providecommand{\ceil}[1]{\left\lceil#1\right\rceil}
\providecommand{\floor}[1]{\left\lfloor#1\right\rfloor}
\renewcommand{\and}{\mathbin{\&}}
\renewcommand{\implies}{\Rightarrow}
\renewcommand{\impliedby}{\Leftarrow}
\DeclareMathOperator{\sech}{sech}
\DeclareMathOperator{\adj}{adj}
\DeclareMathOperator{\infi}{inf}
\DeclareMathOperator{\supre}{sup}
\DeclareMathOperator{\round}{round}
\DeclareMathOperator{\pascal}{pascal}
\DeclareMathOperator{\dom}{domain}
\DeclareMathOperator{\st}{s.t }
\DeclareMathOperator{\argmax}{arg\ max}
\providecommand{\Resc}{\Res\kleene}
\newtheorem{lemma}{Lemma}
\newtheorem{sublemma}{Sublemma}
\sectionfont{\pagebreak[3]}
\subsectionfont{\pagebreak[2]}
\reservestyle{\csyms}{\mathcal}
\csyms{A,B,C,D}
\mathlig{|}{\mid}
\mathlig{==}{\equiv}
\mathlig{~>}{\rightsquigarrow}
\mathlig{~=}{\approx}
\mathlig{~}{\lnot}
\mathlig{|->}{\mapsto}
\mathlig{||-}{\Vdash}
\mathlig{<F>}{\langle{}F \rangle}
\mathlig{|/=}{\nvDash}
\mathlig{[|}{\llbracket}
\mathlig{|]}{\rrbracket}

\title{H\&K Chapters 9, 10}
\author{Sohum Banerjea, Andrew Hedding, Lisa Hofmann}
\date{\today}


\begin{document}

\maketitle

\section{Bound and referential pronouns}
\subsection{Deixis/Anaphora vs Reference/Binding}
\begin{itemize}
\item Describing pronoun behaviour
  \begin{enumerate}
  \item (After John has left the room) I am glad he\textsuperscript{deictic, referring} is gone.
  \item No one likes John. Let's not invite him\textsuperscript{anaphoric, referring}.
  \item Every man is to clean his\textsuperscript{anaphoric, bound} uniform.
  \end{enumerate}
\item These are clearly different categorisations of pronoun behaviour
\item But referring/bound is a useful \emph{semantic} categorisation, in a way deixis/anaphora is not.
\item Bound pronouns use coindexing at syntax and are thus bound semantically in the generated LF
\item Referring pronouns work via salience, and are thus free variables in the generated LF
  \begin{itemize}
  \item (whether salience is extralinguistic or because the anaphora has been recently mentioned)
  \end{itemize}
\end{itemize}

\subsection{Referring pronouns}
\begin{itemize}
\item Referring pronouns thus need the utterance context to provide their variables for the LF to be
  well-defined
\item Capture this by saying that $g$ needs to define any free variables in the LF (``appropriateness'')
\item You can even capture pronoun features, by extending $D_e$ and $D_t$ with failure (call it $\#$) and the
  language with conditioning

  \Tree[. [.{[third person]} ] [. [.{[feminine]} ] [. [.{[singular]} ] [.{she\textsubscript{1}} ] ] ] ]
\item $[|1|] = [|\lambda x.\ \lambda y.\ x|]$ and $[|0|]= [|\lambda x.\ \lambda y.\ y|]$ (Church encoding of booleans)
\item $[feminine] ~> \lambda x_e.\ \textsc{female}(x)\ x\ \#$
\end{itemize}

\section{Pronouns and Ellipsis}
\begin{itemize}
\item In this section, H \& K describe two types of ellipsis: VP Ellipsis and Bare Argument Ellipsis (aka 'stripping')
\item VP ellipsis involves deleting a VP that has an identical antecedent. For example:
\end{itemize}
\Tree 		[.S
				[.S
					[.Laura ]
					[.S$\prime$
						[.$\lbrack$\textsc{past}$\rbrack$ ]
						[.VP
							[.leave ]
							[.Texas ]
						]		
					]
				]	
				[.S 
					[.but ]
					[.S	
						[.Lena ]
						[.S$\prime$
							[.didn't ]
							[.$\langle$VP$\rangle$
								[.$\langle$leave$\rangle$ ]
								[.$\langle$Texas$\rangle$ ]
							]
						]
					]
				]
			]		
\begin{itemize}
\item Ellipsis requires an identical form at LF, and the antecedent licenses optional deletion of the duplicated VP on the PF branch (indicated by $\langle$$\rangle$ in the trees in this handout). 
\item Crucially, the elided material must have an identical LF representation, not only an identical surface string. 
\item Elided material cannot be interpreted with a different scope reading than it's antecedant.
\begin{itemize}
\item Laura showed a drawing to every teacher, but Lena didn't.
\end{itemize}	
\end{itemize}
			
\Tree 	[.S
			[.S
				[.{the milk} ]
				[.S$\prime$
					[.\emph{1} ]
					[.S
						[.Laura ]
						[.VP
							[.drank ]
							[.\emph{t_1} ]
						]	
					]		
				]
			]		
			[.S	
				[.{or perhaps} ]
				[.S
					[.{the juice} ]
					[.S$\prime$
						[.\emph{1} ]
						[.$\langle$S$\rangle$
							[.$\langle$Lena$\rangle$ ]
							[.$\langle$VP$\rangle$
								[.$\langle$drank$\rangle$ ]
								[.$\langle$\emph{t_1}$\rangle$ ]
							]
						]
					]
				]
			]
		]				
\section{Syntactic and semantic binding}

\subsection{Chapter overview}

\subsection{Weak crossover}



\end{document}

%%% Local Variables:
%%% mode: xetex
%%% TeX-master: t
%%% TeX-engine: xetex
%%% End:
